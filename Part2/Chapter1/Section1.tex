% !TEX root = /workspaces/Cookbook/main.tex

\chapter{Why not just use QFT?}
For what it is worth, effective field theory is a powerful tool in the hands of a physicist. It is a way of organizing the information we have about a system in a way that is useful for our purposes. But why bother? Why not just stick to the known Lagrangians and be done with it? The answer is simple: to avoid losing time and energy.\\
When working on the unknown, theoretical particle physics becomes a game of guessing. We guess the Lagrangian, we guess the symmetries, we guess the interactions. And we guess the right ones by comparing the results of our calculations with the experimental data. But what if we are wrong? What if the Lagrangian we guessed is not the right one? Then we have to start all over again.\\
Since one has to be chosen by the particle physics gods to win this game, it is better to have a strategy that minimizes the number of guesses. This is where effective field theory comes in. By using effective field theory, we can make educated guesses about the Lagrangian of a system in low energies, and then test it against the experimental data. If it is wrong, we can modify it in a systematic way, without having to start from scratch.\\
In particular, the interactions regarding a hypothetical particle with the standard model particles would be provable in accelerators, meaning very high energies are out of question. Furthermore, there is a high chance that the low energy phenomena is affected by the 'master' Lagrangian encapsulating some sort of high to low energy symmetry breaking. Since we haven't even observed the fantasy particles yet (othervise they wouldn't be called fantasy), we can simply ignore the high energy part, and construct VERY general toy models, using the so called \textbf{effective} Lagrangians, that are hopefully valid in lower energies.\\
This is the main motivation behind the theory of the effective fields, and we will be discussing the details in the following sections.